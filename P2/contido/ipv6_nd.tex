\chapter{IPv6-Neighbor Discovery}
\label{chap:ipv6_nd}

\section{Ejercico 4.1}
\subsection{¿Por qué es necesario el NS que se envía en t = 6 s? Muestra una captura del log del nodo que lo envía que muestre el motivo del envío. ¿Qué paquete cumple la misma función en IPv4?}

\begin{figure}[H]
    \centering
    \includegraphics[width=135mm, scale=0.75]{imaxes/captura_ejer4_1.png}
    \caption{Paquete NS en t=6s}
    \label{fig:paquete_ns_t6}
\end{figure}

Como se puede ver en la imagen \ref{fig:paquete_ns_t6}, el nodo que envia el paquete Neighbor Solicitation en t=6s es el host[0]. El host[0] empieza la conexión mandando un SYN al servidor remoto pero como no tiene información de su MAC, el router manda al servidor un NS para asi averiguar su MAC. Esto ocurre ya que como se mencinó antes, al no existir una entrada en la Neighbor Cache de host[0] para el serverRemote, el dispositivo debe enviar un mensaje NS para solicitar esa información a través del protocolo Neighbor Discovery.


En el caso de IPv4, el mensaje que realiza una función similar al NS (Neighbor Solicitation) de IPv6 es el mensaje de solicitud ARP (ARP Request). Tanto el NS en IPv6 como la solicitud ARP en IPv4 se utilizan para descubrir la dirección física (de la capa de enlace) que corresponde a una dirección IP o de red específica.

\section{Ejercico 4.2}
\subsection{¿Por qué no se usa una IP unicast para ese mensaje, si ya es conocida?}

No se utiliza la direccion unicast aunque sea conocida ya que en IPv6 se emplea una dirección de multicast especial para la resolución de direcciones. Esto se hace para evitar enviar el mensaje a todos los dispositivos de la red. Esta dirección multicast es escuchada solo por el dispositivo que posee la dirección de destino, limitando el tráfico de resolución de direcciones a los dispositivos interesados y asi no sobrecargar la red, ganando asi eficiencia y recursos.


\section{Ejercico 4.3}
\subsection{Muestra capturas de la neighbor cache del nodo que envió el NS un segundo antes de enviarlo, justo después de enviarlo pero antes de recibir el paquete de respuesta y justo después de recibir la respuesta y explica las diferencias entre los 3 estados. (Nota: para ver la neighbor cache haz doble click sobre el nodo → ipv6 → neighbourDiscovery, y a continuación expande owned objects en la ventana inferior izquierda. La neighbor cache es el atributo neighbourMap.)}


\begin{enumerate}
    \item Un segundo antes de enviar el NS
    
    \begin{figure}[H]
        \centering
        \begin{lstlisting}
if=101 fe80::8aa:ff:fe00:6 ==> 0A-AA-00-00-00-06 ROUTERDefaultRtr STALE reachabilityExp:0 rtrExp:1803.967555297358
        \end{lstlisting}
        \caption{Neighbor cache ante de enviar NS}
        \label{fig:cache_antes_ns}
    \end{figure}

    En este momento, la neighbor cache tiene una entrada para la dirección FE80::8AA:FF:FE00:6, que pertenece a la interfaz del router que está conectado en la red. Esta entrada está en estado STALE, lo que significa que la dirección de enlace (MAC) es conocida, pero la conexión no ha sido verificada recientemente (pasó tiempo cierto tiempo desde que recibió un paquete de esa dirección).

    \item Justo después de enviar el paquete NS
    
 %   \colorbox{yellow}{%
 %           if=101 aaaa:6:65:0:aa54:18ff:fed4:e5f6 ==> 00-00-00-00-00-00 INCOMPLETE reachabilityExp:0\\
 
 %}\\
%  \colorbox{yellow}{%
%        \parbox{\linewidth}{%
%            if=101 fe80::8aa:ff:fe00:6 ==> 0A-AA-00-00-00-06 ROUTERDefaultRtr STALE reachabilityExp:0\\
%            rtrExp:1804.173601335969%
 %       }
  %  }


  \begin{figure}[H]
    \centering
    \begin{lstlisting}
if=101 fe80::8aa:ff:fe00:6 ==> 0A-AA-00-00-00-06 ROUTERDefaultRtr DELAY reachabilityExp:0 rtrExp:1804.173601335969    
    \end{lstlisting}
    \caption{Neighbor cache ante de enviar NS}
    \label{fig:cache_antes_ns}
\end{figure}

   % Llegado a este punto, justo después de enviar el NS, el nodo ha creado una entrada en estado INCOMPLETE para la dirección de destino (AA:6:65:0:AA54:18FF:FED4:E5F6) porque aún no ha recibido respuesta con la dirección MAC correspondiente. La entrada INCOMPLETE indica que el nodo está esperando una respuesta para completar la asociación entre la dirección IPv6 y la dirección MAC. La anterior entrada del neighbor cache permanece en estado STALE.%
   Este estado refleja que el router ha enviado un mensaje NS debido a la expiración de su temporizador de accesibilidad (reachabilityExp:0). El DELAY indica que está en espera de una respuesta antes de pasar a retransmitir el NS, asegurando que la dirección MAC y la accesibilidad de su vecino están actualizadas para mantener la comunicación en la red.

    \item Un segundo después de enviar el paquete NS
    
    \colorbox{yellow}{%
            if=101 aaaa:6:65:0:8aa:ff:fe00:6 ==> 0A-AA-00-00-00-06 STALE reachabilityExp:0%
    }\\
    \colorbox{yellow}{%
        \parbox{\linewidth}{%
            if=101 aaaa:6:65:0:aa54:18ff:fed4:e5f6 ==> A8-54-18-D4-E5-F6 REACHABLE reachabilityExp:36.000003336%
        }
    }\\
    \colorbox{yellow}{%
        \parbox{\linewidth}{%
            if=101 fe80::8aa:ff:fe00:6 ==> 0A-AA-00-00-00-06 ROUTERDefaultRtr STALE reachabilityExp:0\\
            rtrExp:1804.173601335969%
        }
    }

    if=101 aaaa:6:65:0:8aa:ff:fe00:6 ==> 0A-AA-00-00-00-06 STALE reachabilityExp:0

    if=101 fe80::8aa:ff:fe00:6 ==> 0A-AA-00-00-00-06 ROUTERDefaultRtr DELAY reachabilityExp:0 rtrExp:1804.173601335969

    Un segundo después de enviar el NS, el nodo ha recibido la respuesta de la dirección AAAA:6:65:0:AA54:18FF:FED4:E5F6, actualizando la entrada en la neighbor cache al estado REACHABLE y registrando la dirección MAC (A8:54:18:D4:E5:F6). Esto significa que la conectividad con este nodo ha sido verificada recientemente y la información es válida para su uso inmediato. La dirección AAAA:6:65:0:8AA:FF:FE00:6 y la del router (FE80::8AA:FF:FE00:6) se mantienen en STALE, ya que no se ha verificado su conectividad recientemente.

\end{enumerate}



\section{Ejercico 4.4}
\subsection{¿Envía el host[1] algún otro mensaje NS después del instante t = 6 s? ¿Cuál es su objetivo? Muestra una captura del mensaje en Wireshark y una captura del log en la que se muestre el motivo del envío.}


\section{Ejercico 4.5}
\subsection{¿Es la IP destino de este NS del mismo tipo que en los NS enviados anteriormente? ¿Por qué?}


\section{Ejercico 4.6}
\subsection{Muestra capturas de la neighbor cache del host[1] en los siguientes instantes de tiempo:}

\begin{enumerate}
    \item 7 segundos antes del envío del NS de la pregunta anterior.
    
    \item 3 segundos antes del envío del NS.
    \item Justo después del envío del NS y antes de recibir el NA respuesta.
    \item Justo después de recibir el NA respuestas
\end{enumerate}
