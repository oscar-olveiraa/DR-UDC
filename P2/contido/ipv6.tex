\chapter{IPv6}
\label{chap:ipv6}

\section{Ejercicio 2.1}
\subsection{Muestra una captura del escenario en el momento inicial de la simulación en la que se vean todas las
direcciones MAC e IP. Utilizando la dirección IP de host[0] como ejemplo, explica cómo se construye, destacando
los campos y bits relevantes. Utiliza para explicarlo la notación IPv6 no abreviada (16 bytes:
xxxx:xxxx:xxxx:xxxx:xxxx:xxxx:xxxx:xxxx)}

<<<<<<< HEAD
Testeo
=======
\begin{figure}[!ht]
    \centering
    \includegraphics[width=135mm, scale=0.75]{imaxes/captura_ejer2_1.png}
    \caption{Escenario inicial dispositivos con IPv6}
    \label{fig:direccion_ipv6_host0}
\end{figure}

En el principio de la simulación, la direcciones que se muestran son direcciones unicast local de enlace. Estas son direcciones se generan automáticamente una vez que un dispositivo se conecta a una red. La estructura de este tipo de direccione empiezan por FE80:: y este tipo de direcciones no se enrutan.

La segunda parte de la dirección se forma a partir de la dirección MAC del dispositivo. Para esto, fijamos el séptimo bit a 1 e insertamos 0XFFFE entre las dos mitades de la dirección MAC del dispositivo. 

Como ejemplo, vamos a ver la dirección unicast local de enlace que genera el dispositivo host[0]. Como vemos en la imagen \ref{fig:direccion_ipv6_host0}, su dirección está formada como primera parte FE80:: y la segunda parte como 2E8A:21FF:FE7A:8B9C. Esa segunda parte, como se explicó antes, se forma a partir de su dirección MAC (2C-8A-21-7A-8B-9C). Como vemos, en los primeros 2 bytes de la dirección unicast local de enlace (2E8A), si lo comparamos con su dirección MAC, pasa a ser la segunda una E ya que tenemos que añadir un uno en el séptimo bit (la letra C hexadecimal, en binario es 1100, como su tercer bit corresponde al séptimo bit de la dirección unicast, pasa a ser 1 por lo que se convirte en E -> 1110). Llegados a este punto y al haber hecho el primer cambio, tenemos la siguinte estructura -> FE80::2E80:21

Como se explicó anteriormente, una vez hecho el primer cambio, ahora añadimos 0XFFFE  y posteriormente los últimos 3 bytes de la dirección MAC, por lo que queda como dirección final unicast local de enlace FE80::2E8A:21FF:FE7A:8B9C


>>>>>>> 5f9a113 (ejercicio 2_2 ipv6)

\section{Ejercicio 2.2}
\subsection{Asigna al host[2] la misma dirección MAC que al host[0] y arranca la simulación. ¿Qué error ocurre antes de
que haya transcurrido el primer segundo de simulación? Muestra una captura del error que aparece. ¿Qué
paquete (tipo, origen y destino) provoca el error? ¿Por qué?}

\begin{figure}[!ht]
    \centering
    \includegraphics[width=135mm, scale=0.75]{imaxes/captura_ejer2_2.png}
    \caption{Fallo en la red con MAC host[2] igual a la de host[0]}
    \label{fig:direccion_ipv6_host0}
\end{figure}

\section{Ejercicio 2.3}
\subsection{Cambia la MAC del host[2] de manera que coincida con la de host[0] en los últimos 3 bytes y difiera en los 3
primeros bytes (mantén esta MAC para el resto de las cuestiones). Asigna a serverremote la misma MAC que a
host[0]. ¿Vuelve a ocurrir el error de dirección duplicada con serverremote y host[0]? ¿Por qué?}

\section{Ejercicio 2.4}
\subsection{¿Cuánto tiempo transcurre desde el principio de la simulación hasta que el host[0] su IP link-local definitiva
(i.e., fin de DAD)? Muestra la tabla de interfaces del nodo host[0] en la que se vea su estado antes y después del
DAD timeout y explica qué cambia. (Nota: Qtenv muestra toda la información de cada interfaz en una línea;
para verla correctamente copia el contenido con botón derecho → Copy Value y pégalo en la memoria como
texto, en lugar de usar capturas de pantalla.)}

\section{Ejercicio 2.5}
\subsection{¿En qué instante de la simulación obtienen los equipos sus direcciones IP globales? ¿Cómo obtienen esta
última? Muestra la tabla de interfaces de nodo host[0] en la que se vea su estado antes y después de obtener la
dirección global y explica qué cambia.}

\section{Ejercicio 2.6}
\subsection{Explica cómo se construye la IP global usando el nodo host[0] como ejemplo, de nuevo usando la notación
IPv6 no abreviada}

\section{Ejercicio 2.7}
\subsection{Configura host[0] para que se conecte al servidor server remote usando su dirección fe80::x:x:x:x:
(asegúrate de que la dirección MAC de server remote es única). ¿Qué ocurre? Repite lo mismo para
server local. ¿Qué ocurre?}