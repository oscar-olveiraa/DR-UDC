\chapter{IPv6}
\label{chap:ipv6}

\section{Ejercicio 2.1}
\subsection{Muestra una captura del escenario en el momento inicial de la simulación en la que se vean todas las
direcciones MAC e IP. Utilizando la dirección IP de host[0] como ejemplo, explica cómo se construye, destacando
los campos y bits relevantes. Utiliza para explicarlo la notación IPv6 no abreviada (16 bytes:
xxxx:xxxx:xxxx:xxxx:xxxx:xxxx:xxxx:xxxx)}

\section{Ejercicio 2.2}
\subsection{Asigna al host[2] la misma dirección MAC que al host[0] y arranca la simulación. ¿Qué error ocurre antes de
que haya transcurrido el primer segundo de simulación? Muestra una captura del error que aparece. ¿Qué
paquete (tipo, origen y destino) provoca el error? ¿Por qué?}

\section{Ejercicio 2.3}
\subsection{Cambia la MAC del host[2] de manera que coincida con la de host[0] en los últimos 3 bytes y difiera en los 3
primeros bytes (mantén esta MAC para el resto de las cuestiones). Asigna a serverremote la misma MAC que a
host[0]. ¿Vuelve a ocurrir el error de dirección duplicada con serverremote y host[0]? ¿Por qué?}

\section{Ejercicio 2.4}
\subsection{¿Cuánto tiempo transcurre desde el principio de la simulación hasta que el host[0] su IP link-local definitiva
(i.e., fin de DAD)? Muestra la tabla de interfaces del nodo host[0] en la que se vea su estado antes y después del
DAD timeout y explica qué cambia. (Nota: Qtenv muestra toda la información de cada interfaz en una línea;
para verla correctamente copia el contenido con botón derecho → Copy Value y pégalo en la memoria como
texto, en lugar de usar capturas de pantalla.)}

\section{Ejercicio 2.5}
\subsection{¿En qué instante de la simulación obtienen los equipos sus direcciones IP globales? ¿Cómo obtienen esta
última? Muestra la tabla de interfaces de nodo host[0] en la que se vea su estado antes y después de obtener la
dirección global y explica qué cambia.}

\section{Ejercicio 2.6}
\subsection{Explica cómo se construye la IP global usando el nodo host[0] como ejemplo, de nuevo usando la notación
IPv6 no abreviada}

\section{Ejercicio 2.7}
\subsection{Configura host[0] para que se conecte al servidor server remote usando su dirección fe80::x:x:x:x:
(asegúrate de que la dirección MAC de server remote es única). ¿Qué ocurre? Repite lo mismo para
server local. ¿Qué ocurre?}