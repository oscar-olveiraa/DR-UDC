\chapter{Router con QoS - WRR[1,1]}
\label{chap:conqoswrr11}

\section{Longitud de cola del router}

\subsection{Ejercicio 2.1.1}

\renewcommand{\theenumi}{\alph{enumi}}

La cola EF, que representa el flujo UDP, es manejada en este caso mediante \textit{Strict Priority Queueing},
por lo que podemos 'ignorar' las colas de paquetes UDP y simplemente calcular la tasa de salida con la
unica restricción que supone limitar el tráfico EF a la tasa efectiva resultante de transmitir VoIP que, 
en este caso es de 76,8kbps, como se indica en el enunciado.

\[
\label{eq:voip_tasa_entrada_VoIP}
R_{InVoIP} = \frac{1~\text{pkt}}{0,02~\text{s}} = 50 ~ \text{pkt/s}
\]

\[
\label{eq:voip_tasa_salida_VoIP}
R_{outVoIP} = \frac{76,8~\text{kb/s} \cdot 1000~\text{b/kb} \cdot \frac{1~\text{B}}{8~\text{b}}}{ 192~\text{B/pkt}} = 50~\text{pkt/s}
\]

\[
\label{eq:voip_comparacion_tasas_VoIP}
R_{InVoIP} = R_{outVoIP}
\]

Como la tasa de salida es igual a la de entrada, habrá como mucho un paquete VoIP en
cola en todo momento, ya que cada paquete se transmite al mismo tiempo que llega el siguiente.

\vspace{0,3cm}

\subsection{Ejercicio 2.1.2}
\begin{enumerate}
    \item Tasa de salida (en pkt/s y b/s) de cada cola AF1x y AF2x:
    \[
        \label{eq:udp_tasa_salida_con_VoIP}
        \begin{aligned}
            R_{\text{OutUDP}}[b/s] &= R_{\text{Out}} - R_{\text{OutVoIP}} = 128~\text{kb/s} \cdot 1000~\text{b/kb} - 50~\text{pkt/s} \cdot 199~\text{B/pkt} \cdot 8~\text{b/B} \\
                              &= 48400~\text{b/s} \\
            R_{\text{OutUDP}}[pkt/s] &= 48400~\text{b/s} \cdot \frac{1~\text{B}}{8~\text{b}} \cdot \frac{1~\text{pkt}}{1000~\text{B}} = 6,05~\text{pkt/s} \\ \\
            R_{\text{OutAf1}}[b/s] &= p_{\text{Af1}} \cdot R_{\text{OutUDP}} = \frac{1}{2} \cdot 48400~\text{b/s} = 24200~\text{b/s} \\
            R_{\text{OutAf1}}[pkt/s] &= p_{\text{Af1}} \cdot R_{\text{OutUDP}} = \frac{1}{2} \cdot 6,05~\text{pkt/s} = 3,025~\text{pkt/s} \\ \\
        \end{aligned}
    \]
    \item Paquetes por segundo descartados a la entrada de cada cola:
    \[
        \label{eq:udp_paquetes_descartados_con_VoIP}
        \begin{aligned}
            R_{\text{InAfX}} &= \frac{1~\text{pkt}}{0,08~\text{s}} = 12,5~\text{pkt/s} \\ \\
            Pkt_{\text{DescAf1}} &= R_{\text{InAf1}} - R_{\text{OutAf1}} = 12,5~\text{pkt/s} - 3,025~\text{pkt/s} = 11,636~\text{pkt/s} \\
        \end{aligned}
    \]
    \item Tiempo de llenado de las colas AF1x y AF2x:
    \[
        \label{eq:udp_llenado_colas_con_VoIP}
        \begin{aligned}
            t_{\text{FillAf1}} &= \frac{L}{R_{\text{InAf1}} - R_{\text{OutAf1}}} = \frac{100~\text{pkt}}{12,5~\text{pkt/s} - 3,025~\text{pkt/s}} = 9,475~\text{s} \\
        \end{aligned}
    \]

\end{enumerate}

\vspace{0,3cm}

\subsection{Ejercicio 2.1.3}
\[
    \label{eq:udp_tasas_sin_VoIP}
    \begin{aligned}
        R_{\text{InUDP}} &= \frac{1~\text{pkt}}{0,08~\text{s}} = 12,5~\text{pkt/s} \\ \\
        R_{\text{OutUDP}}[pkt/s] &= 128~\text{kb/s} \cdot 1000~\text{b/kb} \cdot \frac{1~\text{B}}{8~\text{b}} \cdot \frac{1~\text{pkt}}{1000~\text{B}} = 16~\text{pkt/s} \\
		R_{\text{OutAf1}}[pkt/s] &= p_{\text{Af1}} \cdot R_{\text{OutUDP}} = \frac{1}{2} \cdot 16~\text{b/s} = 8~\text{pkt/s} \\
    \end{aligned}
\]

\vspace{1cm}

\section{Tiempo en cola del router}

\subsection{Ejercicio 2.2.1}
\[
    \label{eq:udp_tiempo_en_cola_con_VoIP}
    \begin{aligned}
        t_{\text{qAfx}} &= \frac{L}{R_{\text{OutAfx}}} = \frac{100~\text{pkt}}{3,025~\text{pkt/s}}= 33,058~\text{s} \\
    \end{aligned}
\]

\vspace{0,3cm}

\subsection{Ejercicio 2.2.2}
\[
    \label{eq:udp_tiempo_en_cola_sin_VoIP}
    \begin{aligned}
        t_{\text{qAfx}} &= \frac{L}{R_{\text{OutAfx}}} = \frac{100~\text{pkt}}{8~\text{pkt/s}}= 12,5~\text{s} \\
    \end{aligned}
\]

\vspace{1cm}

\section{Retardo extremo a extremo}

\subsection{Ejercicio 2.3.1}
La explicación es la misma que para el caso del router sin QoS aplicada. Las gráficas de \textit{end-to-end delay}
y tiempo de encolado son prácticamente idénticas debido a que, en el caso de esta práctica, el único segmento del 
sistema donde hay congestión y del que surjen todos los problemas es la conexión router-servidor. Además, el sistema
con el que se trabaja es bastante pequeño, haciendo que apenas se note el tiempo de viaje de los paquetes del origen 
a destino. Por tanto, es evidente ver que la práctica totalidad del retraso de los paquetes proeda del tiempo de 
espera en la cola con la que tratamos en estos ejercicios.

\vspace{1cm}

\section{Muestras VoIP perdidas y Paquetes VoIP perdidos}

\subsection{Ejercicio 2.4.1}